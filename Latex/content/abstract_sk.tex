Diplomová práca sa venuje návrhu a implementácií nového spôsobu rýchleho riešenia optimalizačných úloh z oblasti prediktívneho riadenia. Metodológia je založená na distribúcií optimalizačnej úlohy na menšie úseky. V prípade prediktívneho regulátora to znamená rozdelenie na jednotlivé kroky predikčného horizontu. Následne sme využili distribuovaný optimalizačný algoritmus s možnosťou paralelného riešenia. Metódu sme aplikovali na prediktívnom riadení s lineárnym aj nelineárnym modelom systému. Praktická časť obsahuje overenie teórie na simuláciách s cieľom zistiť, či takýmto spôsobom vieme zrýchliť výpočtový čas lineárneho aj nelineárneho regulátora. Záver tejto práce je venovaný návrhu aplikácie, pomocou ktorej môžeme využívať daný spôsob riadenia. Aplikáciu sme rozdelili na dve časti. V prvej časti sme vytvorili server, ktorý nám umožňuje vývoj celej webovej aplikácie. Ten tiež slúži na prácu s údajmi a na vytváranie distribuovaného optimalizačného problému. Druhá časť je prístupná používateľom a nachádza sa vo webovom rozhraní. Pomocou nej môže používateľ definovať parametre riadeného systému a prediktívneho regulátora. Na pozadí tohoto rozhrania sa vykonáva numerická optimalizácia, ktorá slúži pre optimalizáciu distribuovaného optimalizačného problému. Takto navrhnutá aplikácia podporuje paralelné riešenie na viacerých výpočtových zariadeniach. \\
\textbf{Kľúčové slová:} prediktívny regulátor, distribuovaný optimalizačný algoritmus, webová aplikácia, numerická optimalizácia