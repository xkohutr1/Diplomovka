\chapter{Záver}
Cieľom tejto práce bolo navrhnúť, implementovať a overiť nový spôsob rýchleho riešenia optimalizačných problémov z oblasti prediktívneho riadenia. Metóda bola založená na rozdelení predikčného horizontu na menšie úseky s následným využitím distribuovaného optimalizačného algoritmu s možnosťou paralelného riešenia. Záver tejto práce sme venovali vývoju aplikácie, pomocou ktorej je možné využívať takýto spôsob riadenia s decentralizovaným MPC. 

V teoretickej časti sme si rozobrali formáciu predikatívneho regulátora založeného na lineárnom a nelineárnom modeli. Ďalej sme si vo všeobecnosti vysvetlili spôsob distribuovania optimalizačnej úlohy a jej následné riešenie pomocou ADMM, ktorá predstavovala algoritmus pre riešenie daného optimalizačného problému. V závere teórie sme si podrobne vysvetlili spôsob, akým využívame ADMM pri lineárnom a nelineárnom MPC. Pri lineárnom riadení sme použili analytickú optimalizáciu, kde sme položili gradient decentralizovanej účelovej funkcie (rozšírenej Lagrangianovej funkcie) rovný nule a vyjadrili sme si jednotlivé optimalizované premenné. Takýmto spôsobom sme získali všeobecné analytické vzorce, ktoré môžeme priamo využívať v ADMM. Pri komplikovanejšom nelineárnom riadení sme zvolili numerickú optimalizáciu, konkrétne Newtonovu metódu, ktorá slúži pre nájdenie minima decentralizovanej účelovej funkcie (rozšírenej Lagrangianovej funkcie). V tejto práci sme využívali aj ohraničenú optimalizáciu, ktorú sme ale pretransformovali do neohraničenej účelovej funkcie s logaritmickou bariérou.  

V praktickej časti sme si overili teóriu na diskrétnych simuláciách s cieľom zistiť, či takýmto spôsobom vieme zrýchliť výpočtový čas MPC. Simuláciu sme vykonali na dvoch systémoch s lineárnym a nelineárnym modelom. Lineárny model predstavoval zjednodušený pohyb hmotného bodu v priestore. Na základe simulácie sme prišli k záveru, že sa nám podarilo zrýchliť výpočtový čas linearného MPC. Nelineárny model reprezentoval skutočné zariadenie, konkrétne DC motor. Toto zariadenie je systém s veľmi rýchlou dynamikou $Ts = 50ms$. Preto nebolo cieľom dosiahnuť len lepší výpočtový čas ako pri centralizovanom nelineárnom MPC, ale museli sme ho znížiť natoľko, aby sa stíhal vypočítať akčný zásah v rámci periódy vzorkovania riadeného systému. To sa nám pomocou ladenia ADMM podarilo a značne sme zrýchlili výpočtový čas nelineárneho MPC. Môžeme prehlásiť, že takýto spôsob riadenia je možné využívať pre nelineárne systémy s rýchlou dynamikou. 

Ako posledné sme sa venovali vývoju aplikácie, pomocou ktorej vieme využívať decentralizované MPC s možnosťou paralelného riešenia na viacerých výpočtových zariadeniach. Vytvorili sme server v programovacom jazyku Python pomocou knižnice Flask. Server má viacero funkcií, slúži na prácu s dátami uloženými v databázach, vytvára distribuovaný optimalizačný problém z informácií od používateľa a prebieha na ňom diskrétna simulácia. Na strane klienta máme webové rozhranie. Pomocou tohoto rozhrania vie používateľ zadefinovať MPC a ovládať simuláciu. Má možnosť zastaviť simuláciu, alebo ju sledovať na dvoch grafoch v reálnom čase. Výsledky z jednotlivých numerických optimalizácií sú k dispozícií v dynamickej tabuľke. Táto optimalizácia prebieha na pozadí webového prehliadača v programovacom jazyku JavaScript (výpočtový klient). V závere sme vykonali experiment, ktorý potvrdil funkčnosť aplikácie a považujeme ju za správne navrhnutú podľa teórie. 

V budúcnosti by sa aplikácia dala vylepšiť tak, aby podporovala aj iný druh riadenia, ako je stavové MPC, alebo by bolo možné nahradiť simuláciu skutočným zariadením. 