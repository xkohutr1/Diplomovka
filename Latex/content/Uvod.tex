\chapter{Úvod}
\label{ch:uvod}

Vo svete málokedy narazíme na také priemyselné odvetvie, v ktorom by sme sa zaobišli bez automatizácie. Či už sa objaví vo forme automatických pohonov alebo jednoduchých logických spínačov automatizácie je a vždy bude súčasťou priemyslu. A zároveň ako je automatizácia neoddeliteľnou súčasťou priemyslu, je aj riadenie procesov súčasťou, bez ktorej by sme sa v automatizácií nezaobišli. Vďaka riadiacim systémom vieme ovládať a ovplyvňovať komplikované procesy ako sú rektifikácie, chemické reaktory, dokonca aj autopilot v aute či lietadle je len bezpečne implementovaný riadiaci systém, ktorý dohliada na hladký chod zariadenia. Kedysi bolo hlavnou úlohou riadiacich systémov dohliadať na správny a bezpečný chod procesu. Avšak s novou dobou sa začalo dostávať do popredia dôležité šetrenie energií, efektivita procesu a mnohé ďalšie. S vývojom technológií a náročnejšími požiadavkami od technológov sa stali riadiace systémy komplikované a jednoduché PID regulátory neboli už schopné zvládať také náročné úlohy. Stalo sa nevyhnutným, aby boli zahrnuté optimalizácie k riadiacim systémom a spresnili tak výpočty. Najčastejšie využívaným riadením s optimalizáciou sa stalo takzvané prediktívne riadenie, založené na modeli, pod skratkou MPC z anglického (Model Predictive Control). Táto metóda zahŕňa matematický model riadeného systému, pomocou ktorého si vieme vypočítať budúce správanie sa procesu niekoľko krokov dopredu (predikčný horizont). Veľkou výhodou takéhoto riadenia sú ohraničenia optimalizácie, v ktorých môžme zahrnúť jednotlivé požiadavky a obmedzenia, dané správaním sa systému. Avšak všetky tieto úlohy musí MPC vedieť riešiť opakovane v malých časových intervaloch a to môže byť problematické. Niekedy je nutné zanedbať isté predpoklady, nájsť vhodnú metódu, riešenie optimalizácie alebo zvoliť malý predikčný horizont, aby sme zrýchlili výpočtový čas MPC a mohli ho tak využiť pre riadenie systému. 

V tejto práci sa budeme venovať implementácií a overeniu novej metódy rýchleho riešenia optimalizačných problémov z oblasti prediktívneho riadenia a cieľom bude znižovanie výpočtového času MPC. Použitá metóda bude založená na rozložení predikčného horizontu na menšie úseky s následným využitím distribuovaného optimalizačného algoritmu. To umožní rozdeliť optimalizačnú úlohu na menšie časti a riešiť ich samostatne, čo povedie k skráteniu času riešenia, ale aj k zvýšeniu pravdepodobnosti dosiahnutia globálneho optima v nekonvexných úlohách.

V teoretickej časti si predstavíme prediktívne riadenie s lineárnym a nelineárnym modelom. V druhej časti teórie si predstavíme optimalizačnú metódu, ktorú využijeme na riešenie a distribuovanie optimalizačného problému. Následne si teóriu overíme na simuláciách v programovacom jazyku Matlab a porovnáme získané údaje s bežne využívaným MPC na systémoch s rýchlou dynamikou. 

Poslednú časť tejto práce venujeme návrhu aplikácie, vďaka ktorej budeme môcť preniesť teóriu do reálneho sveta. Aplikácia bude rozdelená na dve časti častí:
\begin{enumerate}
	\item Serverová časť (Python),
	\item Klientska časť (HTML, CSS, JavaScript).
\end{enumerate}
V programovacom jazyku Python si vytvoríme server, ktorý bude slúžiť na spracovávanie informácií, ich ukladanie a vytváranie optimalizačných úloh. Druhá časť aplikácie bude vo webovom rozhraní, pomocou jazyka HTML, CSS a JavaScript knižnice JQuery si bude môcť klient zadefinovať riadený systém a parametre MPC. A výhradne pomocou jazyka JavaScript sa bude na strane klienta vykonávať numerická optimalizácia. 

Záverom tohoto projektu bude overená teória nového spôsobu rýchleho riešenia optimalizačných problémov z oblasti prediktívneho riadenia na simulácii so systémom s rýchlou dynamikou a aplikácia pomocou, ktorej bude možné využívať takýto spôsob riadenie. 